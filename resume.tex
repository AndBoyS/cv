%!TEX program = xelatex
%!TEX TS-program = xelatex
%!TEX encoding = UTF-8 Unicode
% Awesome CV LaTeX Template
%
% This template has been downloaded from:
% https://github.com/posquit0/Awesome-CV
%
% Author:
% Claud D. Park <posquit0.bj@gmail.com>
% http://www.posquit0.com
%
% Template license:
% CC BY-SA 4.0 (https://creativecommons.org/licenses/by-sa/4.0/)
%


%%%%%%%%%%%%%%%%%%%%%%%%%%%%%%%%%%%%%%
%     Configuration
%%%%%%%%%%%%%%%%%%%%%%%%%%%%%%%%%%%%%%
%%% Themes: Awesome-CV

\documentclass[]{awesome-cv}
\usepackage{tikz}
\usepackage{ulem}

\newcommand{\ExternalLink}{%
    \tikz[x=1.2ex, y=1.2ex, baseline=-0.05ex]{%
        \begin{scope}[x=1ex, y=1ex]
            \clip (-0.1,-0.1)
                --++ (-0, 1.2)
                --++ (0.6, 0)
                --++ (0, -0.6)
                --++ (0.6, 0)
                --++ (0, -1);
            \path[draw,
                line width = 0.5,
                rounded corners=0.5]
                (0,0) rectangle (1,1);
        \end{scope}
        \path[draw, line width = 0.5] (0.5, 0.5)
            -- (1, 1);
        \path[draw, line width = 0.5] (0.6, 1)
            -- (1, 1) -- (1, 0.6);
        }
    }
\usepackage{textcomp}
\usepackage{hyperref}
%%% Override a directory location for fonts(default: 'fonts/')
\fontdir[fonts/]

%%% Configure a directory location for sections
\newcommand*{\sectiondir}{resume/}

%%% Override color
% Awesome Colors: awesome-emerald, awesome-skyblue, awesome-red, awesome-pink, awesome-orange
%                 awesome-nephritis, awesome-concrete, awesome-darknight
%% Color for highlight
% Define your custom color if you don't like awesome colors
\colorlet{awesome}{awesome-red}
%\definecolor{awesome}{HTML}{CA63A8}
%% Colors for text
%\definecolor{darktext}{HTML}{414141}
%\definecolor{text}{HTML}{414141}
%\definecolor{graytext}{HTML}{414141}
%\definecolor{lighttext}{HTML}{414141}

%%% Override a separator for social informations in header(default: ' | ')
%\headersocialsep[\quad\textbar\quad]
    \begin{document}

%%%%%%%%%%%%%%%%%%%%%%%%%%%%%%%%%%%%%%
%     Profile
%%%%%%%%%%%%%%%%%%%%%%%%%%%%%%%%%%%%%%
\begin{center}
	\headerlastnamestyle{Бахматов} \headerfirstnamestyle{Андрей Вячеславович} \\
	\vspace{2mm}
\end{center}


\vspace{2mm}
\cvsection{Персональная информация}
\begin{cventries}
	\cventry
	{}
	{\def\arraystretch{1.5}{\begin{tabular}{ l  l }
		День рождения:  & {\qquad\skill{17.12.1998}} \\
		Телефон:  & {\qquad\skill{7 (999) 854-98-34}} \\
		Почта:      &{\qquad\skill{\href{mailto:andboyxd@gmail.com}{andboyxd@gmail.com}}    \ExternalLink} \\
		Telegram:      &{\qquad\skill{\href{https://t.me/memento_or}{@memento\_or}}    \ExternalLink} \\
		GitHub:      &{\qquad\skill{\href{https://github.com/AndBoyS/}{@AndBoyS}}    \ExternalLink} \\
		\end{tabular}}}
	{}
	{}
	{}
\end{cventries}

\vspace{-9mm}
%%%%%%%%%%%%%%%%%%%%%%%%%%%%%%%%%%%%%%
%     Education
%%%%%%%%%%%%%%%%%%%%%%%%%%%%%%%%%%%%%%
\cvsection{Образование}
\begin{cventries}
	\cventry
	{Бакалавриат, ПМиИТ, Прикладная математика и информатика}
	{Финансовый университет при Правительстве Российской Федерации}
	{Москва}
	{2017 – 2021}
	{Анализ данных и принятие решений в экономике и финансах}
	\cventry
	{Магистратура, ФКН, Прикладная математика и информатика}
	{Национальный исследовательский университет "Высшая школа экономики"}
	{Москва}
	{2021 - 2023}
	{Финансовые технологии и анализ данных}
\end{cventries}

\vspace{-1mm}

%%%%%%%%%%%%%%%%%%%%%%%%%%%%%%%%%%%%%%
%     Experience
%%%%%%%%%%%%%%%%%%%%%%%%%%%%%%%%%%%%%%

\cvsection{Навыки}
\vspace{-3mm}
\begin{cventries}
	\cventry
	{}
	{\def\arraystretch{1.15}{\begin{tabular}{ l l }
		Python:  & {\qquad\skill{код, тесты, асинхронность, стандартный DS-стек (NumPy, pandas, scikit-learn, PyTorch)}} \\
        Machine Learning:  & {\qquad\skill{Использование большого количества типов классических и глубоких моделей}} \\
		& {\qquad\skill{Опыт решения разнообразных задач:}} \\
		& {\qquad\skill{предсказание дефолтов на графах, определение повреждений автомобиля в computer-vision}} \\
		& {\qquad\skill{Использование передовых моделей deep learning: рекуррентные графовые нейронные сети,}} \\
		& {\qquad\skill{transformer-based универсальные модели сегментации}} \\
        MLOps:  & {\qquad\skill{Уверенное использование git (interactive rebase, squash, worktrees и т.д.)}} \\
		& {\qquad\skill{CLI, работа на Linux через ssh, Docker}} \\
		Работа в команде: & {\qquad\skill{вводил и менторил стажеров и младших разработчиков}} \\
		& {\qquad\skill{отвечал за доведение задач создания моделей от идеи до имплементации}} \\
		Английский: & {\qquad\skill{B2 / Upper-Intermediate}} \\
		\end{tabular}}}
	{}
	{}
	{}
\end{cventries}
\vspace{5mm}
\cvsection{Проекты}
\begin{cventries}
	\cventry
	{Предобратка датасетов, подготовка и реализация графовых НН на PyG-Temporal и DGL, оптимизация архитектуры и гиперпараметров графовых НН}
	{Эмбеддинги компаний в рамках контракта с Газпромбанком}
	{}
	{2021}
	{}
    \cventry
    {Использование, обучение и модификация transformer-based модели сегментации повреждений, создание синтетических датасетов для обучения модели сегментации деталей}
    {Определение повреждений авто по фото из страхового случая в рамках контракта с РЕСО}
    {}
    {2022-2023}
    {}
	\cventry
    {}
    {Преподавание предметов в университете}
    {}
    {2023-2024}
    {Deep Learning \newline Машинное обучение в семантическом и сетевом анализе}
\end{cventries}
\vspace{5mm}
\newpage
\cvsection{Работа}
\begin{cventries}
    \cventry
    {ЦИАРС}
    {Специалист по машинному обучению}
    {}
    {2022 - 2024}
    {}
	\cventry
    {Финансовый университет}
    {Ассистент}
    {}
    {2022 - 2024}
    {}
	\cventry
    {Лаборатория искусственного интеллекта и сетевого анализа, Финансовый университет}
    {Data Scientist}
    {}
    {2021 - 2022}
    {}
\end{cventries}

\end{document}